\documentclass[12pt]{article}

%------------------------------------PACKAGES
\usepackage{graphicx}
\usepackage{setspace}
\usepackage{listings}
\usepackage{xcolor}
\usepackage{amsmath}

%------------------------------------DOCUMENT BEGIN
\begin{document}
\title{OSLab Homework 9}
\maketitle

%-------------------------------------TEXT
\section{Text section}
here is a sample text, as the first part of the homework.\\
we're gonna have 4 more sections for an image, a table, formulas, and a programming code.\\
\begin{center}
   \textbf{{\large LETS GET STARTED!!}} 
\end{center}

\noindent
{\color{red} \rule{\linewidth}{0.5mm} }

%-------------------------------------IMAGE
\section{Image section}
\begin{center}
    here's an image of a fox:\\
    \includegraphics[scale=1]{fox.jpg}
\end{center}

%-------------------------------------TABLE
\section{Table section}
\begin{table}[h]
    \begin{center}
         here's a simple table: \\
         \setstretch{1}
        \begin{tabular}{||c|c|c|c||}
            \hline
            1 & 2 & 3 & 4\\
            \hline
            one & two & three & four\\
            \hline
            first & second & third & forth\\
            \hline
        \end{tabular}
    \end{center}
\end{table}
%------------------------------------FORMULA
\section{Formula section}
    here are some formulas:\\
    (they don't make any sense though)\\
\begin{center}
     \setstretch{2} 
    {\Large
        $(\alpha+\beta)^{2} = \frac{(\sum_{n=1}^{\infty} 2^{-n})^\sigma} {(\lim_{\mu\to\infty} f(x)\times f(y))}$
       $$ A = \frac{\pi a^2}{2} = \sin^2(\alpha)+\cos^4(\alpha) $$
       $$ \iint_A N(a,b) \, da \, db $$
    } 
\end{center}

%-------------------------------------CODE
\newpage
\section{Code Section}
now let's see how we can have a few lines of code:\\
\begin{lstlisting}[language=C++]
    #include <iostream>
    using namespace std;
    int main()
    {    
        int divisor, dividend, quotient, remainder;
        cout << "Dividend:";
        cin >> dividend;
        cout << "Divisor:";
        cin >> divisor;
        quotient = dividend / divisor;
        remainder = dividend % divisor;
        cout << "Quotient=" << quotient << endl;
        cout << "Remainder=" << remainder;
        return 0;
    }
\end{lstlisting}


\end{document}
